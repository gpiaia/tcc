\chapter{Revisão Bibliográfica}

No presente capítulo, são apresentados conceitos básico para a compreensão do presente trabalho. Nessa etapa, busca-se através da revisão de trabalhos já elucidados a contextualização do projeto proposto.

%%%%%%%%%%%%%%%%%%%%%%%%%%%%%%%%%%%%%%%%%%%%%%%%%%%%%%%%%%%%%%%%%%%%%%
\section{Satélites Artificiais} %ch 1303

Desde o lançamento do satélite Sputnik I, primeiro satélite artificial, que foi desenvolvido e lançado pela extinta USSR no dia 4 de Outubro de 1957, muitas técnicas e novas formas de controle foram desenvolvidas. Como o principal fim de um satélite é a comunicação e a navegação, que exigem orientação acurada, juntamente com todas as forças que atuam sobre o satélite, se torna imprescindível o uso de técnicas de controle para se garantir a translação e a rotação do satélite \cite{Brown2002}. Na figura \ref{fig:rotational_brown_p256}, podemos ver os movimentos de rotação e translação de um satélite ao redor de um planeta.

\begin{figure}[!ht]
  \caption{Movimento de Translação de um satélite Artificial.}
  \begin{center}
      \includegraphics[scale=0.5]{img/rotational_brown_p256}
  \end{center}
  \fonte{Adaptado de \citeonline{Brown2002}.} 
  \label{fig:rotational_brown_p256}
\end{figure}

%%%%%%%%%%%%%%%%%%%%%%%%%%%%%%%%%%%
\subsection{Dinâmica de um Satélite}

%%%%%%%%%%%%%%%%%%%%%%%%%%%%%%%%%%%
\subsubsection{Dinâmica de Translação}

A dinâmica de translação de um copo em relação ao outro é um caso de \textit{Dinâmica de Corpo Rígido}. A fundação da Dinâmica de corpos rígidos foi feita pelo Físico Inglês Isaac Newton em forma de três leis, das quais, a 2º e a 3ª são primordiais para descrever a dinâmica de rotação \cite{Snider}. A 2ª lei de Newton pode ser escrita das seguintes forma:

\begin{equation} \label{eq:fma}
  \vec{F}=\frac{d}{dt}\vec{p}\quad\quad\quad\quad\vec{p}=m\vec{v}
\end{equation}

\begin{equation}
  \vec{F}=m\vec{a} 
\end{equation}

Onde $\vec{F}$ é a força, $\vec{p}$ é o momento linear, $\vec{v}$ é a velocidade e $\vec{a}$ é  aceleração.


\begin{equation}\label{eq:ILED}
\vec{F_p}=\vec{f_{ie}}+\sum_{i\neq j}^{N}{\vec{f_{ie}}} = m_i\vec{a_i}
\end{equation}


\begin{equation}
  \sum_{i=1}^{N}{\vec{F_i}}=\sum_{i=1}^{N}{\vec{f_i}}+\sum_{i=1}^{N}{\sum_{i\neq j}^{N}{\vec{f_i}j=}}\sum_{i=1}^{N}{m_i\vec{a_i}} 
\end{equation}

\begin{equation}
  \sum_{i=1}^{N}{\vec{F_i}}=\sum_{i=1}^{N}{\vec{f_i}}+\sum_{i=1}^{N}{\sum_{i\neq j}^{N}{\vec{f_i}j=}}\sum_{i=1}^{N}{m_i\vec{a_i}} 
\end{equation}

\begin{equation}
  \vec{F_{e}}=\sum_{i=1}^{N}{\vec{f_{ie}}}=\sum_{i=1}^{N}{m_i\vec{a_i}} 
\end{equation}

\begin{figure}[!ht]
  \caption{Representação de $\vec{r_{com}}$}
  \begin{center}
      \includegraphics[scale=0.5]{img/translacao_referencial_snider_p15}
  \end{center}
  \fonte{Adaptado de \citeonline{Snider}.} 
  \label{fig:translacao_referencial_snider_p15}
\end{figure}

\begin{equation}
  \vec{r_{com}}=\frac{1}{_MT}\sum_{i=1}^{N}{m_i\vec{r_i}} 
\end{equation}

\begin{equation}
  M_T\frac{{d}^{2}}{{d}t^{2}}\vec{r_{com}}=\sum_{i=1}^{N}{m_i\vec{a_i}}
\end{equation}

\begin{equation}
\vec{F_e}=M_{T}\frac{{d}^{2}}{{d}t^{2}}\vec{r_{com}} 
\end{equation}

%%%%%%%%%%%%%%%%%%%%%%%%%%%%%%%%%%%
\subsubsection{Dinâmica de Rotação}

\begin{figure}[!ht]
  \caption{Distribuição de Massa}
  \begin{center}
      \includegraphics[scale=0.5]{img/mass_snider_p16}
  \end{center}
  \fonte{Adaptado de \citeonline{Snider}.} 
  \label{fig:mass_snider_p16}
\end{figure}

\begin{equation}
  I_{xx}=\sum_{i=1}^{N}{m_i({y_i}^{2}+{z_i}^{2})}
\end{equation}

\begin{equation}
  I_{yy}=\sum_{i=1}^{N}{m_i({x_i}^{2}+{z_i}^{2})}
\end{equation}

\begin{equation}
  I_{zz}=\sum_{i=1}^{N}{m_i({x_i}^{2}+{y_i}^{2})}
\end{equation}

\begin{equation}
  I_{xy}=-\sum_{i=1}^{N}{m_ix_iy_i}=I_{xy}
\end{equation}

\begin{equation}
  I_{xz}=-\sum_{i=1}^{N}{m_ix_iz_i}=I_{zx}
\end{equation}

\begin{equation}
  I_{yz}=-\sum_{i=1}^{N}{m_iy_iz_i}=I_{zy}
\end{equation}

\begin{equation}
I=\begin{bmatrix}I_{xx}&I_{xy}&I_{xz}\\I_{yx}&I_{yy}&I_{yz}\\I_{zx}&I_{zy}&I_{zz}\end{bmatrix}
\end{equation}

Caso o satélite seja simétrico:

\begin{equation}
I=\begin{bmatrix} I_{ xx } & 0 & 0 \\ 0 & I_{ yy } & 0 \\ 0 & 0 & I_{ zz } \end{bmatrix}=\begin{bmatrix} A & 0 & 0 \\ 0 & B & 0 \\ 0 & 0 & C \end{bmatrix}
\end{equation}

\begin{equation}
\omega\equiv\lim_{\Delta t\rightarrow 0}\frac{\Delta\beta}{\Delta t}=\frac{d\beta}{dt}
\end{equation}

\begin{equation}
\alpha\equiv\lim_{\Delta t\rightarrow 0}\frac{\Delta\omega}{\Delta t}=\frac{d\omega}{dt}
\end{equation}

\begin{equation}
d\omega=\alpha dt 
\end{equation}

\begin{equation}
\omega =\omega_{0}-\alpha (t-t_{0})
\end{equation}

\begin{equation}
\beta=\beta_{0}+\omega_{0}(t-t_{0})+\frac{1}{2}\alpha{(t-t_{0})}^{2}
\end{equation}

\begin{equation}
{\omega}^{2}={\omega_0}^{2}+2\alpha(\beta-\beta_0) 
\end{equation}

\begin{equation}
\vec{\tau}=\dot{\vec{L}}
\end{equation}

\begin{equation}
\vec{L}=I\vec{\omega}
\end{equation}

\begin{equation}
\vec{\tau}=I\dot{\vec{\omega}}=I\vec{\alpha}
\end{equation}

\begin{equation}
\dot{\vec{L}}=I\dot{\vec{\omega}}+\vec{\omega}\times I \vec{\omega}
\end{equation}

\begin{equation}
\vec{\tau}=I\dot{\vec{\omega}}+\vec{\omega}\times I \vec{\omega}
\end{equation}

\begin{equation}
\left|\vec{\omega}\right|=\begin{bmatrix}\omega_1\\\omega_2\\\omega_2\end{bmatrix}
\end{equation}

\begin{equation}
\begin{bmatrix}\tau_{1}\\\tau_{2}\\\tau_{3}\end{bmatrix}=\begin{bmatrix}A\dot{\omega}_{1}\\B\dot{\omega}_{2}\\C\dot{\omega}_{3}\end{bmatrix}+\begin{bmatrix}\omega_{1}\\\omega_{2}\\\omega_{3}\end{bmatrix}\times\begin{bmatrix}A\omega_{1}\\B\omega_{2}\\C\omega_{3}\end{bmatrix}
\end{equation}

\begin{equation}
  \tau_1=A\dot{\omega_1}-(B-C)\omega_2\omega_3 
\end{equation}

\begin{equation}
  \tau_2=B\dot{\omega_2}-(C-A)\omega_1\omega_3 
\end{equation}

\begin{equation}
  \tau_3=C\dot{\omega_3}-(A-B)\omega_1\omega_2
\end{equation}

\begin{equation}
\tau_{1}=I_{xx}\dot{\omega_{1}}-I_{xy}(\dot{\omega_{2}}-\omega_{1}\omega_{3})-I_{xz}(\dot{\omega_{3}}+\omega_{1}\omega_{2})\\-(I_{yy}-I_{zz})\omega_{2}\omega_{3}-I_{yz}({\omega_{2}}^{2}-{\omega_{3}}^{2})
\end{equation}

\begin{equation}
 \tau_{1}=I_{xx}\dot{\omega_{1}}-I_{xy}(\dot{\omega_{2}}-\omega_{1}\omega_{3})-I_{xz}(\dot{\omega_{3}}+\omega_{1}\omega_{2})\\-(I_{yy}-I_{zz})\omega_{2}\omega_{3}-I_{yz}({\omega_{2}}^{2}-{\omega_{3}}^{2})
\end{equation}

\begin{equation}
  \tau_{2}=I_{zz}\dot{\omega_{3}}-I_{yz}(\dot{\omega_{3}}-\omega_{1}\omega_{2})-I_{xy}(\dot{\omega_{1}}+\omega_{2}\omega_{3})\\-(I_{zz}-I_{xx})\omega_{1}\omega_{3}-I_{xz}({\omega_{3}}^{2}-{\omega_{1}}^{2})
\end{equation}

\begin{equation}
 \tau_{3}=I_{zz}\dot{\omega_{1}}-I_{xz}(\dot{\omega_{1}}-\omega_{2}\omega_{3})-I_{yz}(\dot{\omega_{2}}+\omega_{1}\omega_{3})\\-(I_{xx}-I_{yy})\omega_{1}\omega_{2}-I_{xy}({\omega_{1}}^{2}-{\omega_{2}}^{2})
\end{equation}

\begin{equation}
  \dot{\omega_{1}}=\frac{\tau_{1}}{A}+\left(\frac{B-C}{A}\right)\omega_{2}\omega_{3}
\end{equation}

\begin{equation}
  \dot{\omega_{2}}=\frac{\tau_{2}}{B}+\left(\frac{C-A}{B}\right)\omega_{1}\omega_{3}
\end{equation},

\begin{equation}
  \dot{\omega_{3}}=\frac{\tau_{2}}{C}+\left(\frac{A-B}{C}\right)\omega_{1}\omega_{2}
\end{equation}

\subsubsection{Matriz de Rotação}

\begin{equation}
  cos\phi=\hat{i_{1}}\cdot\hat{b}
\end{equation}

\begin{equation}
  cos\theta=\hat{i_{2}}\cdot\hat{b}
\end{equation}

\begin{equation}
  cos\varphi=\hat{i_{3}}\cdot\hat{b}
\end{equation}

\begin{figure}[!ht]
  \caption{Distribuição de Massa}
  \begin{center}
      \includegraphics[scale=0.5]{img/coordenadas_snider_p25}
  \end{center}
  \fonte{Adaptado de \citeonline{Snider}.} 
  \label{fig:coordenada}
\end{figure}

\begin{equation}
R^{11} = \hat{i_{1}}\cdot\hat{b_1}
\end{equation}

Onde a matriz de rotação 
\begin{equation}
 R^{ib}=\begin{bmatrix}R_{11}&R_{12}&R_{13}\\R_{21}&R_{22}&R_{23}\\R_{31}&R_{32}&R_{33}\end{bmatrix}
\end{equation}

\begin{figure}[!ht]
  \caption{Ângulos de Euler}
  \begin{center}
      \includegraphics[scale=0.5]{img/euler_snider_p28}
  \end{center}
  \fonte{Adaptado de \citeonline{Snider}.} 
  \label{fig:coordenada}
\end{figure}


\begin{equation}
R_{ij}=\hat{i_{3}}\cdot\hat{b}=cos\beta_{ij} \\
\begin{Bmatrix} 1-2-3 & \quad 2-3-1 & \quad 3-1-2 \\ 1-3-2 & \quad 2-1-3 & \quad 3-2-1 \\ 1-3-1 & \quad 2-1-2 & \quad 3-2-3 \end{Bmatrix} 
\end{equation}

\begin{equation}
R_{3}(\phi)\leftarrow R_{2}(\theta)\leftarrow R_1(\psi)\\
\end{equation}

\begin{equation}
R_{3}(\phi)R_{2}(\theta)R_{1}(\psi)=\begin{bmatrix}1&0&0\\0&cos\phi&sen\phi\\0&-sen\phi&cos\phi\end{bmatrix}\begin{bmatrix}cos\theta&0&-sen\theta\\0&1&0\\-sen\theta&0&cos\theta\end{bmatrix}\begin{bmatrix}cos\psi&sen\psi&0\\-sen\psi&cos\psi&0\\0&0&1\end{bmatrix}
\end{equation}

\begin{equation}
R(\phi ,\theta ,\psi )=\begin{bmatrix} cos\theta cos\psi  & cos\phi sen\theta  & -sen\theta  \\ -cos\phi sen\psi +sen\phi sen\theta cos\psi  & cos\phi cos\psi +sen\phi sen\theta sen\psi  & sen\phi cos\theta  \\ sen\phi sen\psi +cos\phi sen\theta cos\psi  & -sen\phi cos\psi +cos\phi sen\theta sen\psi  & cos\phi cos\theta  \end{bmatrix}
\end{equation}

\begin{equation}
R(\phi ,\theta ,\psi )=\begin{bmatrix} 1 & \psi  & -\theta  \\ -\psi  & 1 & \phi  \\ \theta  & -\phi  & 0 \end{bmatrix} \\
\end{equation}

\begin{equation}
\phi=tan^{-1}\left(\frac{R_{23}}{R_{33}}\right)
\end{equation}

\begin{equation}
\theta=tan^{-1}\left(\frac{-R_{13}}{\sqrt{1-R^{2}_{13}}}\right)
\end{equation}

\begin{equation}
\psi=tan^{-1}\left(\frac{R_{12}}{R_{11}}\right) 
\end{equation}

\begin{equation}
\vec{\omega_1}(\dot{\phi})\leftarrow\vec{\omega_2}(\dot{\theta})\leftarrow\vec{\omega_3}(\dot{\psi})
\end{equation}

\begin{equation}
\begin{bmatrix}\omega_{1}\\\omega_{2}\\\omega_{3}\end{bmatrix}=\begin{bmatrix}\dot{\phi}\\0\\0\end{bmatrix}+R_{1}(\phi)\begin{bmatrix}0\\\dot{\theta}\\0\end{bmatrix}+R_{1}(\phi)R_{2}(\theta)\begin{bmatrix}0\\0\\\dot{\psi}\end{bmatrix}
\end{equation}

\begin{equation}
 = \begin{bmatrix} 1 & 0 & -sen\theta \\ 0 & cos\phi  & sen\phi cos\theta \\ 0 & -sen\phi & cos\phi cos\theta\end{bmatrix}\begin{bmatrix}\dot{\phi}\\\dot{\theta}\\\dot{\psi}\end{bmatrix}
\end{equation}

\begin{equation}
\begin{bmatrix} \dot { \phi  }  \\ \dot { \theta  }  \\ \dot { \psi  }  \end{bmatrix}=\frac { 1 }{ cos\theta  } \begin{bmatrix} cos\theta  & sen\phi sen\theta  & cos\phi sen\theta  \\ 0 & cos\phi cos\theta  & -sen\phi cos\theta  \\ 0 & sen\phi  & cos\phi  \end{bmatrix}\begin{bmatrix} \omega_1 \\ \omega_2  \\ \omega_3  \end{bmatrix} 
\end{equation}

\begin{equation}
\begin{bmatrix} \dot { \phi  }  \\ \dot { \theta  }  \\ \dot { \psi  }  \end{bmatrix}=\frac { 1 }{ sin\theta  } \begin{bmatrix} sen\psi  & cos\psi  & 0 \\ cos\psi sen\theta  & -sen\psi sen\theta  & 0 \\ -sen\psi cos\theta  & -cos\psi cos\theta  & 0 \end{bmatrix}\begin{bmatrix} \omega _{ 1 } \\ \omega _{ 2 } \\ \omega _{ 3 } \end{bmatrix}
\end{equation}


\begin{figure}[!ht]
  \caption{Distribuição de Massa}
  \begin{center}
      \includegraphics[scale=0.5]{img/euler_angles_snider_p27}
  \end{center}
  \fonte{Adaptado de \citeonline{Snider}.} 
  \label{fig:coordenada}
\end{figure}

\subsubsection{Rodas de Reação}

\begin{equation}
\vec{L_{tot}}=\vec{L_{b}}+\vec{L_{\omega}}=constante 
\end{equation}

\begin{equation}
\vec{L_{b}}=I\vec{\omega}
\end{equation}

\begin{equation}
\vec {L_{\omega} } =D_{\omega}\vec{\psi_{\omega}} 
\end{equation}

\begin{equation}
\vec{\tau_{\omega}}=\dot{\vec{L_{\omega}}}=D_{\omega}\dot{\vec{\psi_{\omega}}}
\end{equation}

\begin{equation}\label{eq:torquefinal1}
\tau_{1}=A\dot{\omega_{1}}+(D_{\omega}\dot{\psi_{\omega}})_{1}+(C-B)\omega_{2}\omega_{3}-(D_{\omega}\psi_{\omega})_{2}\omega_{3}+(D_{\omega}\psi_{\omega})_{3}\omega_{2}
\end{equation}

\begin{equation}\label{eq:torquefinal2}
\tau_{2}=B\dot{\omega_{2}}+(D_{\omega}\dot{\psi_{\omega}})_{2}+(A-C)\omega_{1}\omega_{3}+(D_{\omega}\psi_{\omega})_{2}\omega_{3}-(D_{\omega}\psi_{\omega})_{3}\omega_{1}
\end{equation}

\begin{equation}\label{eq:torquefinal3}
\tau_{3}=C\dot{\omega_{3}}+(D_{\omega}\dot{\psi_{\omega}})_{3}+(B-A)\omega_{1}\omega_{2}-(D_{\omega}\psi_{\omega})_{1}\omega_{2}+(D_{\omega}\psi_{\omega})_{3}\omega_{1}
\end{equation}

\begin{figure}[!ht]
  \caption{Representação Mecânica Simplificada de um satélite com rodas de Reação.}
  \begin{center}
      \includegraphics[scale=0.75]{img/satellite_controlhand_p1306}
  \end{center}
  \fonte{Adaptado de \citeonline{Levine1996}.} 
  \label{fig:satellite_controlhand_p1306}
\end{figure}



%%%%%%%%%%%%%%%%%%%%%%%%%%%%%%%%%%%%%%%%%%%%%%%%%%%%%%%%%%%%%%%%%%%%%%
%conceitos de resposta em frequência
% estabilidade de sistemas
% em regime
% diferentes tipos de sistemas

%%%%%%%%%%%%%%%%%%%%%%%%%%%%%%%%%%%%%%%%%%%%%%%%%%%%%%%%%%%%%%%%%%%%%%
\section{Controlador PID}

\subsection{Resposta ao Degrau do um Sistema}

Um problema fundamental em engenharia, é prever e modelar sistemas naturais ou artificias para tirarmos o melhor proveito de suas características. Para isso, muitas técnicas foram desenvolvidas para se conseguir controlar essas variáveis de interesse \cite{Levine1996}.

Uma forma clássica de representar um resposta de uma variável de interesse, é através da resposta ao degrau. Essa pode ser vista na figura \ref{fig:transient_ogata_p170}, onde temos as principais características da resposta ao degrau de um sistema de segunda ordem ou superior. Onde \textit{$M_p$} (maximum overshoot) é o sobressinal do da variável de interesse em percentual, \textit{$t_d$} (delay Time) é o tempo atraso de transporte, \textit{$t_r$} (rise time) é o tempo necessário para atingir o valor do sinal de referência, \textit{$t_p$} é o tempo de pico (peak time) e \textit{$t_s$} (settling time) é o tempo para o sistema entrar em regime, observando um critério de erro em regime \cite{Ogata}.

\begin{figure}[!ht]
  \caption{Parâmetros de uma Resposta ao Degrau de um sistema de segunda ordem ou superior.}
  \begin{center}
      \includegraphics[scale=0.5]{img/transient_ogata_p170}
  \end{center}
  \fonte{Adaptado de \citeonline{Ogata}.} 
  \label{fig:transient_ogata_p170}
\end{figure}

A resposta ao degrau nos diz muito sobre os sistemas de interesse, pois nela podemos ver claramente a atuação dos \textit{polos} e \textit{zeros} dominantes e o ganho de baixa frequência da função de transferência do sistema. Os métodos clássicos para se calcular os parâmetros de controladores são baseados na resposta ao degrau \cite{Ogata}. 

%%%%%%%%%%%%%%%%%%%%%%%%%%%%%%%%%%%
\subsection{Controlador PID}

O controlador proporcional-integral-derivativo é um dos controladores mais utilizados nas aplicações industriais. Essa topologia de controlador consegue, em diferentes configurações atender entre 90 e 95\% de todos os sistemas que necessitam de controladores \cite{Levine1996}. A forma descritiva matemática mais comum de se encontrar um controlador PID no domínio do tempo é a seguinte:

\begin{equation}
  u(t) = K\left((e(t)+\frac{1}{T_i}\int_{0}^{t}{e(\tau)}d\tau+T_d\frac{de(t)}{dt}\right) 
\end{equation}

Onde \textit{e(t)} é o erro, \textit{$T_i$} é o tempo integral, \textit{$T_d$} é o tempo derivativo e \textit{K} o ganho proporcional. Uma outra forma de se representar os tempos integral e derivativo, é através dos ganhos \textit{$K_i$} que é o ganho integral e o \textit{$K_d$}, que é o ganho derivativo \cite{Astrom1995};

A partir desse princípio podemos relacionar com as equações \ref{eq:torquefinal1}, \ref{eq:torquefinal2} e \ref{eq:torquefinal3} podemos escrever a relação do controlador PID do Torque do motor:

\begin{equation}
\tau_{m_{1,2,3}}=K_{P}\left((\beta_{com}-\beta)+\frac{1}{T_{I}}\int{(\beta_{com}-\beta)dt}+T_{d}\frac{d}{dt}(\beta_{com}-\beta)\right)
\end{equation}

\begin{equation}
=K_{P}\left(\beta_{com}+\frac{1}{T_{I}}\int{\beta_{com}dt}+T_{d}\frac{d}{dt}\beta_{com}\right) 
\end{equation}

Na figura \ref{fig:pid_controller_Snider_p35}, podemos ver a configuração mais utilizada do controlador PID, onde $\beta_{com}(S)$ é o valor do sinal de referência no domínio da frequência, \textit{err(S)} o erro, \textit{$K_p$} é o ganho proporcional, \textit{$K_i$} é o ganho integral, \textit{$K_d$} é o ganho derivativo, \textit{$G_e(s)$} é a função de transferência do controlador PID, \textit{$M_{c1}(s)$} é o sinal de erro tratado pelo controlador, \textit{D(S)} é um distúrbio, \textit{$G_p(S)$} é a função de transferência da planta e por fim, \textit{$\beta(S)$} que é a variável de interesse \cite{Snider}.

\begin{figure}[!ht]
  \caption{Representação do Modelo de Controlador PID com distúrbios.}
  \begin{center}
      \includegraphics[scale=0.75]{img/pid_controller_Snider_p35}
  \end{center}
  \fonte{\citeonline{Snider}.} 
  \label{fig:pid_controller_Snider_p35}
\end{figure}

Como alguns sistemas podem admitir grandes e rápidas variações de sinais de referência, exigindo uma grande força de controle e energia no atuador, algumas topologias foram desenvolvidas para limitar a atuação de alguns fatores dos controladores. Um bom exemplo é o \textit{anti-windup}, onde essa topologia limita a saturação do atuador quando o sistema atinge o regime, causado pela ação integral. Essa topologia pode ser vista no modelo da figura \ref{fig:pid_antiwindup_astrom_p83} \cite{Astrom1995}.

\begin{figure}[!ht]
  \caption{Modelo de um Controlador PID com Histerese}
  \begin{center}
      \includegraphics[scale=0.65]{img/pid_antiwindup_astrom_p83}
  \end{center}
  \fonte{\citeonline{Astrom1995}.} 
  \label{fig:pid_antiwindup_astrom_p83}
\end{figure}

Existem muitas combinações de controladores PID, cada uma com suas peculiaridades e vantagens de uso. Uma combinação muito usada é a PI, onde a acão integral de zerar o erro em regime e uma boa resposta transitória já satisfazem as especificações. Podemos ver um exemplo dessa configuração na imagem \ref{fig:pi_twomotors_astrom_p308}, onde apenas um controlador PI controla a velocidade angular (\textit{$\omega$})somada de dois motores (Motor 1 e Motor 2) a partir de uma velocidade de referência (\textit{$\omega_{sp}$}) \cite{Astrom1995}.

\begin{figure}[!ht]
  \caption{Representação do Modelo de Controlador PI com dois motores.}
  \begin{center}
      \includegraphics[scale=0.65]{img/pi_twomotors_astrom_p308}
  \end{center}
  \fonte{Adaptado de \citeonline{Astrom1995}.} 
  \label{fig:pi_twomotors_astrom_p308}
\end{figure}

%%%%%%%%%%%%%%%%%%%%%%%%%%%%%%%%%%%%%%%%%%%%%%%%%%%%%%%%%%%%%%%%%%%%%%
\section{Sintonia de Controladores}

Como podemos ver no modelo matemático e gráfico dos controladores PID, os valores de ajuste $K_p$, $T_i$ e $T_d$  podem assumir infinitos valores, sendo necessário a escolha do melhor conjunto desses valores para que a planta desempenhe o comportamento esperado \cite{Ogata}.


%%%%%%%%%%%%%%%%%%%%%%%%%%%%%%%%%%%
\subsection{Método de sintonia de Ziegler-Nichols}

Dois métodos clássicos de sintonia foram desenvolvidos em 1942 por Ziegler e Nichols. Esses dois métodos são baseados em características da resposta ao degrau e o com a resposta em frequência. Na figura \ref{fig:ziegler-nichols_astrom_p135} podemos ver a resposta ao degrau e os parâmetros de atraso (L) e velocidade da resposta (a), que são usados para o cálculos dos parâmetros do controlador  \cite{Astrom1995}. 

\begin{figure}[!ht]
  \caption{Resposta ao degrau e os parâmetros de atraso (L) e velocidade da resposta (a).}
  \begin{center}
      \includegraphics[scale=0.75]{img/ziegler-nichols_astrom_p135}
  \end{center}
  \fonte{\citeonline{Astrom1995}.} 
  \label{fig:ziegler-nichols_astrom_p135}
\end{figure}

Na tabela \ref{tab:Ziegler-Nichols} podemos ver as relações dos parâmetros $K_p$, $T_i$, $T_d$ e $T_p$ com os vistos na figura \ref{fig:ziegler-nichols_astrom_p135}.  

\begin{table}
  \caption{Parâmetros PID pelo Método de Ziegler-Nichols - Resposta ao Degrau}
  \label{tab:Ziegler-Nichols}
  \centering%
  \begin{minipage}{.42\textwidth}
    \begin{tabular*}{\textwidth}{lllll}
      \hline
      {Controlador} & {K} & {$T_i$} & {$T_d$}& {$T_p$}\\ \hline
      \hline
      P    &  1/a   &     &      & 4L  \\ 
      PI   &  0.9/a & 3L  &      & 5.7L  \\
      PID  &  1.2/a & 2L  & L/2  & 3.4L  \\ \hline
    \end{tabular*}
    \fonte{Adaptado de \citeonline{Astrom1995}}
  \end{minipage}
\end{table}

O outro método de se estimar os valores do controlador, é através de duas características da resposta em frequência: uma delas que é o ganho que deixa o sistema marginalmente estável ($K_u$), e a outra, é o período do sinal de referência que deixa o sistema marginalmente estável ($T_u$). Podemos ver na tabela \ref{tab:Ziegler-Nichols-freq} as relações entre $K_p$, $T_i$, $T_d$, $T_p$ e ($K_u$) e ($T_u$) . 

\begin{table}
  \caption{Parâmetros PID pelo Método de Ziegler-Nichols - Resposta em Frequência}
  \label{tab:Ziegler-Nichols-freq}
  \centering%
  \begin{minipage}{.52\textwidth}
    \begin{tabular*}{\textwidth}{lllll}
      \hline
      {Controlador} & {K} & {$T_i$} & {$T_d$}& {$T_p$}\\ \hline
      \hline
      P    &  0.5$K_u$   &           &             & $T_u$  \\ 
      PI   &  0.4$K_u$   & 0.8$T_u$  &             & 1.4$T_u$ \\
      PID  &  0.6$K_u$   & 0.5$T_u$  & 0.125$T_u$  & 0.85$T_u$  \\ \hline
    \end{tabular*}
    \fonte{Adaptado de \citeonline{Astrom1995}}
  \end{minipage}
\end{table}

%\subsection{Métodos de Otimização ????}
%%%%%%%%%%%%%%%%%%%%%%%%%%%%%%%%%%%%%%%%%%%%%%%%%%%%%%%%%%%%%%%%%%%%%%
\subsection{Sintonia Automática de Controladores e Controle Adaptativo}


Como muitos sistemas sofrem variações temporais, distúrbios e o interesse do usuário em modificar constantemente a resposta da planta, cria-se a necessidade de automatizar o processo de sintonia, ao um simples comando do usuário.

Além da simplicidade do conceito de implementação, os controladores PID são muito utilizados pela possibilidade de auto sintonia e adaptatividade dos controladores, pois a partir do comportamento da resposta ao degrau podemos calcular os parâmetros do controlador e aplicarmos sem a intervenção humana\cite{Astrom1995}. Na figura \ref{fig:pid_adaptative_astrom_p233}, podemos a partir do comportamento da planta, escolher a ou as técnicas mais adequadas que devemos implementar para controlar de forma satisfatória a planta.

\begin{figure}[!ht]
  \caption{Fluxograma para a escolha do Método de Sintonia.}
  \begin{center}
      \includegraphics[scale=0.75]{img/escolha_controle_astrom_p236}
  \end{center}
  \fonte{Adaptado de \citeonline{Astrom1995}.} 
  \label{fig:pid_adaptative_astrom_p233}
\end{figure}


%%%%%%%%%%%%%%%%%%%%%%%%%%%%%%%%%%%
\subsubsection{Método do Relé}

Características da resposta em frequência podem ser descobertas se somarmos ao sinal de referência, uma onda retangular ao mesmo tempo em que o controlador PID está desconectado. Esse é o princípio do método de sintonia usando relé, onde o relé desempenha o papel do chaveamento e cria um sinal retangular sobreposto ao sinal de referência \cite{Levine1996}. Na imagem \ref{fig:pid_autotuning_relay_astrom_p239} podemos ver o conceito básico do método do relé.
 
\begin{figure}[!ht]
  \caption{Modelo do Método de Auto Sintonia via Relé.}
  \begin{center}
      \includegraphics[scale=0.75]{img/pid_autotuning_relay_astrom_p239}
  \end{center}
  \fonte{Adaptado de \citeonline{Astrom1995}.} 
  \label{fig:pid_autotuning_relay_astrom_p239}
\end{figure}

O modelo matemático usado para descrever o comportamento de um relé, pode ser visto na equação da sequência que é obtida através da transformada de Fourier:

\begin{equation}
  N(a)=\frac{4d}{\pi a}\left(\sqrt{1-\left(\frac{\varepsilon}{a}\right)^{2}}-i\frac{\varepsilon}{a}\right) 
\end{equation}

Onde \textit{d} é a amplitude de oscilação do relé (normalmente até 10\% do sinal de referência), \textit{$\varepsilon$} é a histerese do relé e \textit{a} é a amplitude do sinal de referencia \cite{Levine1996}. Isso pode ser visto na figura \ref{fig:relay_signals}.

\begin{figure}[!ht]
  \caption{Sinais do Relé e da Saída do Sistema}
  \begin{center}
      \includegraphics[scale=0.75]{img/relay_giap_p2}
  \end{center}
  \fonte{Adaptado de \citeonline{Giap2014}} 
  \label{fig:relay_signals}
\end{figure}

Com isso, podemos calcular os parâmetros intermediários para a sintonia do controlador da seguinte forma:

\begin{equation}
  K_u \approx \frac{4d}{\pi a} \\
  T_u = P_u
\end{equation}

Onde $P_u$ é o período de oscilação do sinal de interesse. Munido desses valores, podemos recorrer a tabela do método Ziegler-Nichols em resposta em frequência (Tabela \ref{tab:Ziegler-Nichols-freq}).

%%%%%%%%%%%%%%%%%%%%%%%%%%%%%%%%%%%
%\subsubsection{Método Baseado em Regras} %Astron 241


%%%%%%%%%%%%%%%%%%%%%%%%%%%%%%%%%%%%%%%%%%%%%%%%%%%%%%%%%%%%%%%%%%%%%%
\section{Sistemas Embarcados}

\begin{figure}[!ht]
  \caption{Estrutura Básica do Espaço de Usuário e de Kernel.}
  \begin{center}
      \includegraphics[scale=0.75]{img/kernel_user_space}
  \end{center}
  \fonte{Adaptado de Embarcados....} 
  \label{fig:kernel_user_space}
\end{figure}
  
\begin{figure}[!ht]
  \caption{Representação do Modelo de Controlador PID com distúrbios}
  \begin{center}
      \includegraphics[scale=0.35]{img/sistema-linux-overview_embarcados}
  \end{center}
  \fonte{Adaptado de Embarcados....} 
  \label{fig:sistema-linux-overview_embarcados}
\end{figure}


%%%%%%%%%%%%%%%%%%%%%%%%%%%%%%%%%%%
\subsection{Comportamento Assíncrono}

%%%%%%%%%%%%%%%%%%%%%%%%%%%%%%%%%%%
\subsection{Escalonamento de Tarefas}

%%%%%%%%%%%%%%%%%%%%%%%%%%%%%%%%%%%
\subsection{Processamento Digital de Sinais}

\begin{figure}[!ht]
  \caption{Resposta ao degrau com diferentes períodos de amostragem.}
  \begin{center}
      \includegraphics[scale=0.65]{img/nise_digitalinput_p761}
  \end{center}
  \fonte{Adaptado de \citeonline{Nise}.} 
  \label{fig:nise_digitalinput_p761}
\end{figure}

\subsubsection{Protocolos de MQTT e I2C}
%%%%%%%%%%%%%%%%%%%%%%%%%%%%%%%%%%%%%%%%%%%%%%%%%%%%%%%%%%%%%%%%%%%%%%
\section{Estado da Arte}

%%%%%%%%%%%%%%%%%%%%%%%%%%%%%%%%%%%
\subsection{Sistemas Inteligentes de Sintonia de Controladores}


\begin{figure}[!ht]
  \caption{Modelo de um Controlador com Auto-sintonia.}
  \begin{center}
      \includegraphics[scale=0.75]{img/pid_adaptative_astrom_p233}
  \end{center}
  \fonte{Adaptado de \citeonline{Astrom1995}.} 
  \label{fig:pid_adaptative_astrom_p233}
\end{figure}


%%%%%%%%%%%%%%%%%%%%%%%%%%%%%%%%%%%
\subsection{Controle Fuzzy} %Astrom p298


%%%%%%%%%%%%%%%%%%%%%%%%%%%%%%%%%%%
\subsection{Controle com Redes Neurais}  %control hand p1017

\begin{figure}[!ht]
  \caption{Modelo de um Simples Neurônio.}
  \begin{center}
      \includegraphics[scale=0.6]{img/neuron_astrom_p295}
  \end{center}
  \fonte{Adaptado de \citeonline{Astrom1995}.} 
  \label{fig:neuron_astrom_p295}
\end{figure}

\begin{figure}[!ht]
  \caption{Representação de uma Rede Neural.}
  \begin{center}
      \includegraphics[scale=0.65]{img/feedforward_neural_astrom_p297}
  \end{center}
  \fonte{Adaptado de \citeonline{Astrom1995}.} 
  \label{fig:feedforward_neural_astrom_p297}
\end{figure}

\begin{figure}[!ht]
  \caption{Proposta de \citeonline{Chen2004} para um Controlador PID com Redes Neurais}
  \begin{center}
      \includegraphics[scale=1]{img/pid_neural_Applying_p18}
  \end{center}
  \fonte{Adaptado de \citeonline{Chen2004}.} 
  \label{fig:pid_neural_Applying_p18}
\end{figure}

%%%%%%%%%%%%%%%%%%%%%%%%%%%%%%%%%%%%%%%%%%%%%%%%%%%%%%%%%%%

\cite{Verma2018}
\cite{Li2010}
\cite{Nise}
\cite{Chen2004}
\cite{Liguo2008}
\cite{Amaral}
\cite{Araari2014}
\cite{Rolle2009}
\cite{Li}
\cite{Ogata}
\cite{Snider}
\cite{Gohiya2012}
\cite{Hu2001}
\cite{Johnson}
\cite{Editor}
\cite{Li2013}
\cite{Das2017}
\cite{Dorf}
\cite{Levine1996}
\cite{Thomas}
\cite{Guo2010}
\cite{Liu2011}
\cite{Astrom1995}
\cite{Behera2017}
\cite{Brown2002}