% ----------------------------------------------

% ||||||||||||||||||||||||||||||||||||||||||||||
% EXEMPLOS
% ||||||||||||||||||||||||||||||||||||||||||||||

% ----------------------------------------------
%	Figure example
% ----------------------------------------------
%
% \begin{figure}[htp]
% \centering
% \caption{Aqui vai o caption da imagem.}
% \includegraphics[width = 0.9\linewidth ]{images/x.png}
% \label{fig:x}
% \legend{Fonte: Adaptado de \citeonline{x}.}
% \end{figure}

% ----------------------------------------------
%	Figure example
% ----------------------------------------------
%
%\begin{figure}[htp]
%\centering
%\includegraphics[width = 0.5\linewidth ]{dlayer.jpg}
%\caption{\label{fig:1}Complanar waveguide parameters using two dielectric layers \cite{cwccs}.} 
%\end{figure}

% ----------------------------------------------
%	Equation example \ref{eqn:1}
% ----------------------------------------------
%
%\begin{eqnarray}
%\epsilon _{r} & = & \epsilon _{r}'(1-i\tan(\delta))
%\label{eqn:1}
%\end{eqnarray}

% ----------------------------------------------
%	Complex equations example \ref{eqn:2}
% ----------------------------------------------
%
%\begin{eqnarray}
%C_{1} & = & 2\epsilon _{0}(\epsilon _{r1}-1)\frac{K(k_{1})}{K(k_{1}')}\nonumber\\
%& = & 2\epsilon _{0}(\epsilon _{r1}'-i\tan(\delta)\epsilon _{r1}'-1)\frac{K(k_{1})}{K(k_{1}')}\nonumber\\
%& = & 2\epsilon _{0}(\epsilon _{r1}'-1)\frac{K(k_{1})}{K(k_{1}')}-i2\epsilon _{0}(\tan(\delta)\epsilon _{r1}')\frac{K(k_{1})}{K(k_{1}')}\\
%C_{2} & = & 2\epsilon _{0}(\epsilon _{r2}-\epsilon _{r1})\frac{K(k_{2})}{K(k_{2}')}\nonumber\\
%& = & 2\epsilon _{0}(\epsilon _{r2}'-i\tan(\delta)\epsilon _{r2}'-\epsilon _{r1}'+i\tan(\delta)\epsilon _{r1}')\frac{K(k_{2})}{K(k_{2}')}\nonumber\\
%& = & 2\epsilon _{0}(\epsilon _{r2}'-\epsilon _{r1}')\frac{K(k_{2})}{K(k_{2}')}-i2\epsilon _{0}(\tan(\delta)\epsilon _{r2}'-\tan(\delta)\epsilon _{r1}')\frac{K(k_{2})}{K(k_{2}')}\\
%C_{vac} & = & 4\epsilon _{0}\frac{K(k_{0})}{K(k_{0}')}
%\label{eqn:2}
%\end{eqnarray}

% ----------------------------------------------
%	Table example \ref{tab:1} 
% ----------------------------------------------
%
%\begin{table}[htb]
%\caption{Constants and Parameters.}
%\begin{center}
%\begin{tabular}{|c|c|c|c|}
%\hline
%\bfseries CONSTANT & \bfseries VALUE & \bfseries CONSTANT & \bfseries VALUE \\
%\hline \hline
%$\epsilon _{0}$ & 8.8540$\times$10$^{-12}$ & $c$ & 299792458~m/s \\
%\hline
%$\epsilon _{r1}$ & 11.7 & $h_{1}$ & 300~$\mu$m \\
%\hline
%$\epsilon _{r2}$ & 7.5 & $h_{2}$ & 200~nm \\
%\hline
%$\delta _{1}$ & 10$^{-4}$ & $\delta _{2}$ & 10$^{-2}$ $\sim$ 10$^{-3}$ \\
%\hline
%\end{tabular}
%\end{center}
%\label{tab:1}
%\end{table}

% ----------------------------------------------
%	Complex table example \ref{tab:2}
% ----------------------------------------------

%\begin{table}[htb]
%\caption{Results and dimensions.}
%\begin{center}
%\begin{tabular}{|c|c|c|c|c|}
%\hline
%\bfseries $f_{0}$ [MHz] & \bfseries $S$ [$\mu$m] & \bfseries $W$ [$\mu$m] & \bfseries $d$ [cm] & \bfseries $C_{c}$ [fF] \\
%\hline
%\hline
%\multirow{4}{*}{650} & \multirow { 2}{*}{2} & \multirow{ 2}{*}{1} & \multirow{ 2}{*}{9.4966} & 10 \\
%\cline{5-5}
%& & & & 30 \\
%\cline{2-4} \cline{5-5}
%& \multirow { 2}{*}{4.5} & \multirow { 2}{*}{2.3} & \multirow { 2}{*}{9.3155} &         10 \\
%\cline{5-5}
%& & & & 30 \\
%\hline
%\multirow { 4}{*}{6000} &  \multirow { 2}{*}{2} &  \multirow { 2}{*}{1} &  \multirow { 2}{*}{1.0288} &          1 \\
%\cline{5-5}
%& & & & 3 \\
%\cline{2-4} \cline{5-5}
%& \multirow { 2}{*}{4.5} &  \multirow { 2}{*}{2.3} &  \multirow { 2}{*}{1.0092} &          1 \\
%\cline{5-5}
%& & & & 3 \\
%\hline

%\end{tabular}
%\end{center}
%\label{tab:2}
%\end{table}

% ----------------------------------------------
%	Two graphics in one \ref{fig:2}
% ----------------------------------------------
%
%\begin{figure}[htp]
%\centering
%\begin{tabular}{cc}
%(a) & (b) \\
%\includegraphics[width = 0.5\linewidth ]{DM_6G_4p5_3_6.jpg} &
%\includegraphics[width = 0.5\linewidth ]{DM_650_4p5_10_6.jpg}
%\end{tabular}
%\caption{captions} 
%\label{fig:2}
%\end{figure}

% ----------------------------------------------
%	Four graphics in one table \ref{fig:3}
% ----------------------------------------------
%
%\begin{figure}[htp]
%\centering
%\begin{tabular}{cc}
%(a) & (b) \\
%\includegraphics[width = 0.5\linewidth ]{DM_650_4p5_10_1.jpg} &
%\includegraphics[width = 0.5\linewidth ]{DM_650_4p5_30_1.jpg} \\
%(c) & (d) \\
%\includegraphics[width = 0.5\linewidth ]{DM_6G_4p5_1_1A.jpg} &
%\includegraphics[width = 0.5\linewidth ]{DM_6G_4p5_3_1.jpg} \\
%\end{tabular}
%\caption{Captions} 
%\label{fig:3}
%\end{figure}

% ----------------------------------------------
%	Quote and footnote
% ----------------------------------------------
%
%\begin{quote} ``adkfjahsldkfjashflkasdjfhadslkfjhasdlfkjadshflsda
%kjdshflkasjdfhalskfjhadslfksdajhfladskjfhsda
%kajsdfhlaksdjfhasdkl'' \footnote{test}
%\end{quote}

% ----------------------------------------------
%	PDF Annotation
% ----------------------------------------------
%
%\pdfannot % generic annotation
%width 10cm % the dimension of the annotation can be controlled
%height 0cm % via <rule spec>; if some of dimensions in
%depth 4cm % <rule spec> is not given, the corresponding
% value of the parent box will be used.
%{ %
%/Subtype /Text % text annotation
%/Open true % if given then the text annotation will be opened
%/Contents % text contents
%(write comments in here...)
%}%
%

%\chapter{Código em C}
%  \begin{lstlisting}
%  #include <stdio.h>

%  void main(void)
%  {
%  	return 0;
%  }
% \end{lstlisting}




% ||||||||||||||||||||||||||||||||||||||||||||||
% DOCUMENTO PARA ENTREGA VERSÃO FINAL
%||||||||||||||||||||||||||||||||||||||||||||||
% \clearpage\thispagestyle{empty}\addtocounter{page}{-1}
% \section*{ENTREGA DA VERSÃO FINAL DO TRABALHO DE CONCLUSÃO DE CURSO}

% \vspace*{2cm}
% Eu, \textsc{\imprimirorientador}, autorizo o aluno(a) \textsc{\imprimirautor} a entregar a versão final do Trabalho de Conclusão de Curso à secretaria. O trabalho foi por mim analisado e está de acordo com os apontamentos feitos pelos membros da banca de apresentação do referido aluno.

% \vspace*{1cm}
% \begin{center}   
%    \assinatura{{\imprimirorientador} \\ Orientador}
% \end{center}

% \vspace*{1cm}
% \begin{flushright}
% 	\imprimirlocal, 11 de Janeiro de \imprimirdata.
% \end{flushright}
% \clearpage

% ----------------------------------------------




% \begin{table}[!htb]
% \caption{Classificação dos principais distúrbios da qualidade de energia}
% \label{tab-prqe}
% \centering
% \begin{tabular}{p{2cm}p{2cm}p{4cm}p{3cm}p{3cm}}
% \hline
% \bfseries Problemas & \bfseries Categoria & \bfseries Categorização  & \bfseries Causas & \bfseries Efeitos \\
% \hline 
% \multirow{3}{*}{Transientes} 
% & Impulsivo & Pico, tempo de subida e duração & Queda de raios,  energização de transformadores, chaveamento de cargas capacitivas & Ressonância do sistema de energia \\
% & \space & \space & \space & \space \\
% & Oscilatório & Amplitude e frequência dos componentes & Chaveamento da rede, capacitores ou cargas & Ressonância do sistema \\ 

% \hline

% \multirow{5}{2cm}{Variações em tensão de curta duração} 
% & Subtensão \textit{Sag Voltage} & Magnitude,  duração & Partida de motores, curto-circuito monofásico & Incorreto funcionamento de proteções, perda de produção \\
% & \space & \space & \space & \space \\
% & Sobretensão \textit{Swell Voltage} & Magnitude,  duração & Chaveamento de capacitores e grandes cargas,  faltas & Incorreto funcionamento de proteções,  alto estresse em computadores e eletrodomésticos \\ 
% & \space & \space & \space & \space \\
% & Interrupção & Duração & Faltas temporárias & Incorreto funcionamento de sistemas de proteção à incêndio \\
% \hline

% \multirow{3}{2cm}{Variações em tensão de longa duração} 
% & Subtensão \textit{Undervoltage} & Magnitude,  duração & Chaveamento de cargas, desenergização de capacitores & Aumento de perdas e aquecimento \\
% & \space & \space & \space & \space \\
% & Sobretensão \textit{Overvoltage} &Magnitude, duração & Desligamento de cargas, energização de capacitores & Danos a eletrodomésticos\\ 
% \hline
% \multirow{3}{2cm}{Distorções da forma de onda} 
% & \textit{Offset} DC & Tensão e corrente & Distúrbios geomagnéticos e retificação & Saturação em transformadores \\
% & \space & \space & \space & \space \\
% & Harmônicas & THD, harmônicas, espectro & cargas não-lineares & Aumento das perdas, baixo FP \\ 
% & \space & \space & \space & \space \\
% \end{tabular}
% \fonte{Adaptado de \citeonline{book:PQ}.}
% \end{table}