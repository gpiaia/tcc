% ----------------------------------------------
% CONCLUSÃO
% ----------------------------------------------
\chapter[Conclusão]{Conclusão}

Na primeira parte do trabalho, foram abordados os princípios da dinâmica de um satélite, de técnicas clássicas e modernas para a sintonia de controladores. Foram descritos conceitos como rodas de reação, dinâmica de corpo rígido, controladores PID, método do relé, redes neurais artificias, entre outros. O referencial descreve principalmente as possibilidades de sintonia de controladores e a modelagem de um satélite artificial.

A implementação do protótipo foi dividida em 3 partes, uma foi o desenvolvimento do \textit{hardware}, outra o \textit{software}, e ainda o sistema de controle, levando em consideração a melhora na resposta ao degrau e da dinâmica do satélite através da sintonia adequada do controlador. Na análise de resultados, pôde-se observar a melhora na resposta transitória do simulador de satélites, mesmo com perceptível presença de ruídos.

Ainda na análise de resultados, ficou clara a interferência nos resultados pela forte presença de ruídos nos sinais oriundos do giroscópio, provocados pela vibração dos motores e da própria física do giroscópio, pois, mesmo parado, apresenta ruídos estocásticos. Também é perceptível uma oscilação durante o movimento de rotação do satélite, esse movimento indesejado fica cada vez menor ao passo que o sinal de controle é menor, ou seja, uma rápida variação do sinal de controle, faz com que o corpo oscile devido à baixa inércia do simulador em comparação ao torque dos motores.

Através desse trabalho, ficou clara a possibilidade de se empregar técnicas modernas em controle, onde uma RNA aprendeu o comportamento de dinâmica de um simulador de satélites, sendo possível se aplicar em diversas outras plantas. Assim, nesse trabalho, foi possível atingir os objetivos propostos, salvo melhorias que podem ser desenvolvidas em trabalhos futuros.

Trabalhos futuros a partir desse podem ser desenvolvidos, principalmente com a substituição do giroscópio simples por um sensor inercial absoluto. Esse outro sensor conta com magnetômetro, barômetro, acelerômetro, giroscópio e uma unidade que processa os sinais e entrega as grandezas sem comprometer o processamento do controlador, além de uma ótima exatidão e blindagem maior ao ruídos provenientes dos motores. Ainda, é desejável se criar uma interface gráfica para facilitar o usuário operar e configurar o simulador de satélites, com o objetivo de aprendizagem de conceitos de controle. 


