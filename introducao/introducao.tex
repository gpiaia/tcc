% ----------------------------------------------
% INTRODUCAO
% ----------------------------------------------
\chapter[Introdução]{Introdução}

O controle de atitude de satélites é estudado desde o final dos anos 50, quando os primeiros satélites foram colocados em órbita. O controle acurado da posição é necessário para manter o alinhamento com o planeta ou alvo. Isso se dá através de dispositivos que pela variação do momento angular, realizam movimentos de rotação no corpo do satélite . Para solucionar esse problema, diversas técnicas de controle e sintonia já foram empregadas no decorrer das últimas décadas, dentre elas, o controlador PID (Proporcional-Integral-Derivativo) que é usado para controlar a velocidade angular dos motores que são utilizados para o movimento de rotação do corpo do satélite.

Os controladores PID precisam de sintonia, ou seja, dependendo da dinâmica do sistema em que eles estão inseridos, ou ainda, alguma mudança que ocorra no sistema, o controlador precisa ser reajustado para que atenda as especificações previamente estabelecidas pelo projetista. Como os satélites estão inacessíveis e em um ambiente hostil, técnicas automáticas e adaptativas são irrescindíveis para garantir o perfeito funcionamento e a vida útil do satélite. Algumas técnicas de sintonia automática já foram desenvolvidas, algumas usam princípios clássicos de controle e tabelas para especificações engessadas, outras, modernas, fazem o uso de inteligência artificial para buscar os parâmetros ótimos.


Este trabalho tem por principal objetivo a implementação em um sistema embarcado de um controlador com sintonia automática utilizando conceitos de controle inteligente. Essa implementação será embarcada em um dispositivo com Linux  e fará o controle de atitude de um simulador de satélites, que é a planta do sistema de controle. Com a utilização de diferentes técnicas de controle, uma comparação entre as diferentes técnicas será feita, com o objetivo de avaliar de forma quantitativa e qualitativa a implementação do controle inteligente.

O trabalho está organizado em capítulos, onde o primeiro é a presente introdução. O segundo é o desenvolvimento teórico, que aborda os principais conceitos dos elementos básicos para o desenvolvimento do trabalho. Já o terceiro, é a metodologia, onde os materiais, desenhos mecânicos e todos os passos para o desenvolvimento estão descritos. No quarto temos o cronograma, onde uma estimativa de tempo e prazos são apresentados. E por último, as considerações finais sobre as atividades realizadas até a presente etapa do trabalho.











